\chapter{Tomografías de Coherencia Óptica}
Toda la información de este capítulo ha sido cuidadosamente
seleccionada y adaptada de \emph{Avances en la exploración de la
  retina: OCT:\@ Qué es y para qué sirve}\cite{oct-bib}

\section{Imágenes \glsentrytext{tcog}}
Las \gls{tcoa} son un tipo de imágenes ampliamente utilizadas en el
ámbito hospitalario por los oftalmólogos para estudiar, explorar y monitorizar las
estructuras oculares de los pacientes para el diagnóstico de
enfermedades, especialmente, de difícil identificación oftalmológica.

\section{Máquina \glsentrytext{tcog}}
Una máquina \gls{tcoa} se utiliza para la exploración en tiempo real, sin ser
invasiva e indolora, para obtener con resolución micrométrica imágenes
de estructuras de tejidos vivos. Con el paso del tiempo se ha hecho
insustituible en el diagnóstico y control de numerosas patologías
oculares, así como también la mejora de su rapidez y precisión.

\subsection{Propiedades ópticas de los tejidos}
La máquina \gls{tcoa} es capaz de medir y representar los distintos
tipos de tejidos en escala de grises o colores según su reflectividad:
\begin{description}
\item[Reflectividad alta:] sangre, exudados lipídicos, epitelio
  pigmentario, coriocapilar o capas de fibras nerviosas
  perpendiculares al haz de luz. Representados con colores cálidos
  como rojo y blanco.
\item[Reflectividad media:] capas retinianas de la limitante interna a
  la plexiforme externa. Representadas con colores verdes y amarillos
\item[Reflectividad baja:] zonas de edema, vítreo y contenido seroso,
  dispuestos en paralelo al haz de luz. Representados con colores
  fríos como azul y negro.
\end{description}

\subsection{Base teórica}
La máquina \gls{tcoa} usa una técnica análoga a las máquina de
ecografías pero sustituyendo los ultrasonidos por luz. La máquina mide
la latencia o retardo e intensidad de la onda reflejada tras incidir
en un tejido. La ventaja de una máquina \gls{tcoa} frente a una de
ecografías es la velocidad de la luz, muy superior a la velocidad de
los ultrasonidos, del orden de \emph{$10^{-15}$ fentosegundos}, requiriendo un sistema de medición indirecto.\\
\emph{El interferómetro de Michelson} es la principal base del
funcionamiento de estas máquinas. Consiste en dirigir la radiación a
un divisor de haces que a su vez los divide en dos. Un haz se dirige
por un medio conocido mientras que el otro por el medio a
estudiar. Tras esto, ambos haces se reflejan para hacerlos coincidir
en un mismo punto, patrón del que se registra la interferencia entre
ambos para medir la intensidad y retardo del que recorrió el medio
desconocido.

\subsection{Historia}
Estas primeras máquinas operaban sobre el \emph{dominio temporal}, es decir,
el espejo recorría toda la superficie del medio a estudiar para así
reflejar el haz de referencia. La velocidad de traslación del espejo
se convertía en el factor crítico y limitante de la velocidad de
captura, en los mejores casos, \emph{400 barridos por segundo}.\\
La velocidad de barrido es crítica principalmente por dos motivos:
\begin{description}
\item [Disminución del tiempo de estudio] de la estructura ocular. Lo
  que reduce la irritación y la sequedad del ojo del paciente que al
  tener que permanecer abierto durante todo el proceso incide en la
  calidad del resultado obtenido.
\item [Aumento de la resolución y calidad] directamente proporcional
  debido al mayor número de muestras realizadas.
\end{description}
Con el paso del tiempo y el perfeccionamiento de la técnica, se hizo
posible fijar el espejo y aumentar así los barridos por segundo
mediante el uso del \emph{dominio espectral o de Fourier}. Esta
técnica se sustenta en una serie de dispositivos que analizan el haz a
comparar con el de referencia. Con el espejo fijo en un punto y estos
nuevos dispositivos, supuso un avance enorme al permitir \emph{25.000
  barridos por segundo}.\\
Actualmente, las máquinas más recientes trabajan con una variante del
\emph{dominio espectral}, se denominan de tipo
\emph{swept-source}. Utilizan el dominio de la frecuencia codificada
en el tiempo (\emph{Time encoded frequency domain}). Para ello, un
láser sintonizable emite el haz con una longitud de onda fija que
muestrea secuencialmente el medio con una serie de longitudes de onda
individuales que posteriormente descodifica. Así, alcanza un
\emph{100.000 muestreos por segundo}. Además proporciona imágenes muy
precisas y con mayor profundidad mostrando tejidos que anteriormente
había que enfocar en niveles más profundos como la \textbf{coroides} o
menos detalle como el vitreo.

\subsection{Resolución y densidad de muestreo}

\subsection{Número de rastreos en el mismo punto}

\subsection{Software de análisis de la información}

\subsection{Tipos de exploración}

\subsection{Seguimiento}

\subsection{Otras características}

\subsection{Informes}

\section{Aplicaciones clínicas}

\section{Calidad de la explotación y artefactos}
