\chapter{Plan de trabajo}
La estructura de la memoria para mayor claridad se ha dividido en
cinco partes:
\begin{description}
\item[Parte I, Antecedentes:] Esta primera parte está escrita a modo
  de introducción al trabajo, los problemas y los objetivos planteados.
\item[Parte II, Visión computerizada:] La segunda parte se ha
  dedicado a las bibliotecas y técnicas de visión computerizada
  aplicadas en la investigación. Se ha profundizado todo lo
  posible con imágenes de ejemplo usadas durante el estudio después de
  cada explicación.
\item[Parte III, Investigación:] En esta parte se explica
  todo lo referente al trabajo de estudio e investigación realizado.
\item[Parte IV, Propuesta software:] Parte dedicada al código y a los 
  algoritmos propuestos que permiten resolver los problemas planteados 
  y alcanzar los objetivos descritos en la primera parte.
\item[Parte V, Conclusiones:] Finalmente, esta
  quinta y última parte contiene las conclusiones derivadas de la
  investigación, así como un capítulo orientado a temas de estudio
  de cara al futuro: Todas las ideas, las propuestas, los problemas 
  y los objetivos que se han encontrado a lo largo del trabajo que, 
  por falta de tiempo y conocimientos, no se han podido llevar a cabo, 
  pero que sin duda alguna serán muy pronto la base de nuevas investigaciones 
  y trabajos.
\end{description}