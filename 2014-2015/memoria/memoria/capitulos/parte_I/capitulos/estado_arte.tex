\chapter{Estado del arte}
Uno de los primeros puntos que se tuvo en cuenta al elaborar un
trabajo de estas características, fue hacer un análisis del estado del arte.\\
Un estudio del estado del arte consiste en la evaluación del estado de
evolución de las técnicas y tecnologías a investigar. Los resultados
de este proceso fueron muy vagos y escasos. Los oftalmólogos recuerdan
de algún estudio en los últimos años acerca de obtener mayor partido a
las imágenes de \gls{tco} mediante algún tipo de procesado de la
estructura ocular de manera externa e independiente de la máquina \gls{tco}.\\
Gracias a que nuestro tutor, Carlos, puso en nuestro conocimiento una
noticia \emph{Optical modelling of the human +
  retina}\footnote{\url{http://ecmiindmath.org/2015/04/07/optical-modelling-of-the-human-retina/}}
de la \gls{ECMI}, descubrimos que en el \emph{Centro para las
  Matemática de la Universidad de
  Coimbra\footnote{\url{http://www.uc.pt/uid/lcm/projects/currentProjects/retina\_model}}
  (Portugal)}, en un esfuerzo interdisciplinar en el campo de la
\emph{ingeniería biomédica}, aplican las mismas técnicas e ideas de
procesamiento (lógicamente a un nivel de recursos en todos los ámbitos
muy superior a los abarcados aquí) a
las \gls{tco} que dieron lugar a este proyecto. \\
Sin embargo, los objetivos son bien distintos. El objetivo que
persiguen es tan ambicioso como genérico. Quieren desarrollar un
algoritmo que incorpore las ecuaciones de \emph{Maxwell} dependientes
del tiempo a las longitudes de onda para así solucionar el problema de
dispersión para cada una de las capas de la retina de los haces de luz
emitidos por la máquina \gls{tco}. Esto supondría un grandísimo avance
en cuanto a la calidad se refiere, pudiendo dar lugar a una mejora de
la precisión y abrir el estudio de otras estructuras oculares que se
puedan investigar utilizando máquinas \gls{tco}, hasta ahora
difícilmente
visibles.\\
Como ejemplo de aplicación real, se centran en el \emph{edema macular
  diabético}, una complicación de los pacientes con diabetes. Este
edema es la primera causa de ceguera entre este tipo de pacientes.
