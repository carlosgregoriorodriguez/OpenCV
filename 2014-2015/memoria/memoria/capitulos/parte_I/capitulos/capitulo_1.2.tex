\chapter{Tomografías de Coherencia Óptica}
\section{¿Qué son?}
\section{Máquina \glsentrytext{tcog}}
Una máquina \gls{tcoa} se utiliza para la exploración en tiempo real, sin ser
invasiva e indolora, para obtener con resolución micrométrica imágenes
de estructuras de tejidos vivos.Con el paso del tiempo se ha hecho
insustituible en el diagnóstico y control de numerosas patologías
oculares, así como también la mejora de su rapidez y precisión.
\subsection{Base teórica}
La máquina \gls{tcoa} usa una técnica análoga a las máquina de ecografías
pero sustituyendo los ultrasonidos por luz. La máquina mide la
latencia o retardo e intensidad de la onda reflejada tras incidir en
un tejido. La ventaja de una máquina \gls{tcoa} frente a una de ecografías es
la velocidad de la luz, muy superior a la velocidad de los
ultrasonidos requiriendo un sistema de medición indirecto.
\subsection{Propiedades ópticas de los tejidos}
La máquina \gls{tcoa} es capaz de medir y representar los distintos
tipos de tejidos en escala de grises o colores según su reflectividad:
\begin{description}
\item[Reflectividad alta:] sangre, exudados lipídicos, epitelio pigmentario,
coriocapilar o capas de fibras nerviosas perpendiculares al haz de
luz. Representados con colores cálidos como rojo y blanco.
\item[Reflectividad media:] capas retinianas de la limitante interna a la
plexiforme externa. Representadas con colores verdes y amarillos
\item[Reflectividad baja:] zonas de edema, vítreo y contenido seroso,
dispuestos en paralelo al haz de luz. Representados con colores fríos
como azul y negro.
\end{description}
\section{Imágenes \glsentrytext{tcog}}