\chapter{Introduction}
\section{Problem}
This project araises from the need to improve and automatize the
study of images within the area of the medicine, with the goal of 
increasing the diagnosis accuracy and speeding it up. Sometimes, the
machines that take these images lack of important characteristics 
that the doctors may need, or hints to monitor the patient's evolution.

In this project the objects of study are the \emph{OCT} images to monitor
two diseases: \emph{uveitis} and \emph{glaucoma}. Due to that,
it has received the cooperation of ophthalmologists who work at
\emph{Hospital Universitario 12 de Octubre}.

\section{Goals}
The medical goals where so clear to the team of ophthalmologists
we collaborate with that they had no problem to manually indicate on an
image the information they would need to identify from the programs that 
manage the \emph{OCT}. With these requirements, the next goals were defined
to the project:
\begin{itemize}
\item Obtain images that are as close to the manually indicated ones as possible.
\item Automatize the process, or else simplify it to be executed in as few steps as possible.
\item Provide the knowledge gathered from the programs, data and 
dimensions in order to improve the understanding of the \emph{OCT} and
the diseases.
\item Break the dependency from the software of the \emph{OCT} machine
which can not be improved or adapted, due to author rights and patents, 
in order to solve the problems the ophthalmologists have.
\item Learn how to get along in an external area to the computer science
with a great number of unknown techniques, trying to reach usefull 
results which can be applied in medicine.
\item Understand the problem as one, the goals and all the proposes ideas.
This way, in the future, it will be possible to suggest improvements, venture
on new issues and develop more complex techniques to reach more
ambicious goals.
\end{itemize}

