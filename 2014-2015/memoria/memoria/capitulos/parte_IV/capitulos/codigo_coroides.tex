\chapter{Medición de la coroides: Implementación}
\section{Manual de uso}
Debido a la necesidad de uso del algoritmo por parte de los
oftalmólogos, se les ha proporcionado junto al programa para la
medición del espesor de la \emph{\gls{coroides}} el siguiente manual de uso:\\
El programa no requiere de intervención externa desde que se ejecuta
hasta que termina de realizar la medición, sin embargo, para facilitar
su uso, se han implementado los siguientes parámetros de ejecución con
su correspondiente ayuda.
\begin{minted}{text}
  usage: algorithm.py [-h] [-p] (-a ARCHIVOS [ARCHIVOS ...] | -c
  CARPETAS [CARPETAS ...])  algorithm.py: error: one of the arguments
  -a/--archivos -c/--carpetas is required
\end{minted}

Al programa se tiene que especificar obligatoriamente si se
quiere procesar un archivo, varios o una carpeta entera.
\begin{description}
\item[Procesar archivos:] el programa puede procesar tanto un archivo
  como varios archivos uno detrás de otro. Para ello, introducir el
  parámetro \emph{-a} seguido de uno o varios archivos.
 \begin{minted}{bash}
   $ python2 algorithm.py -a tomografia1.png tomografia2.png
  \end{minted}
\item[Procesar carpetas:] de la misma manera que se procesan los
  archivos, se procesan las carpetas. Para ello, introducir el
  parámetro \emph{-c} seguido de una o varias carpetas.
 \begin{minted}{bash}
    $ python2 algorithm.py -c carpeta
  \end{minted}
\end{description}

Además, existen dos modos de procesamiento:
\begin{description}
\item[Con pasos:] para generar una imagen por cada técnica aplicada
  por el algoritmo sobre la imagen. Para ello, introducir el parámetro
  \emph{-p}.
  \begin{minted}{bash}
    $ python2 algorithm.py -p -a tomografia.png
  \end{minted}
\item[Sin pasos:] únicamente genera la imagen resultante con la medida
  del espesor. Opción predeterminada.
  \begin{minted}{bash}
    $ python2 algorithm.py -p -a tomografia.png
  \end{minted}
\end{description}

\section{Código}

\begin{codigo_python}
  \caption{Código de algorithm.py}
  \inputminted[fontsize=\scriptsize, linenos, breaklines]{python}{../OpenCV/algorithm.py}
\end{codigo_python}