\chapter{Medición del área de los poros de la papila}
\section{Problema}
En el curso de la investigación que se está llevando
a cabo actualmente sobre el \emph{glaucoma} y las
causas que lo provocan, se ha detectado
una posible relación con la aparición de porosidades
en la \emph{papila retiniana}. \\
La \emph{papila retiniana} o \emph{disco óptico} 
es el punto donde el nervio óptico entra en el globo 
ocular. Tiene la forma de un pequeño disco rosado de 
aproximadamente 2 x 1.5 milímetros y está situado en 
la parte posterior del globo ocular. Hasta hace 
relativamente poco no se sabía mucho de ella debido 
a las carencias tecnológicas. Gracias a las \gls{tcoa},
se han podido conseguir datos con cierta precisión, 
entre los que se encuentra el objeto de este estudio:
unos poros de los que se cree que puedan estar 
relacionados con la aparición del \emph{glaucoma}.\\

\section{Objetivos}
El objetivo principal de esta parte de la investigación 
implica encontrar automáticamente, mediante algoritmos 
de visión computerizada, los poros del disco óptico
para medirlos y, en la medida de lo posible, establecer
una correlación entre su tamaño, la cantidad, la posición
o la forma y la aparición del \emph{glaucoma}, siendo
este procedimiento supervisado por personal médico.

\section{Estudio}
\section{Algoritmo propuesto}
