\section{Aportes de Miguel Madrid Mencía}
Mis aportes se distribuyen a lo largo de todo el proyecto, desde la
definición de los primeros esbozos del trabajo hasta los últimos
detalles de la memoria.\\
He separado mis aportaciones en dos apartados, uno para todo lo
relacionado con los algoritmos realizados y otro referente al
desarrollo de la memoria. Además un apartado más para los
conocimientos adquiridos durante todo este tiempo.

\subsection{Código}
Para poder realizar los algoritmos propuestos, primero me tuve que
familiarizar no sólo con el lenguaje propuesto, en nuestro caso,
\emph{Python}. Tomamos la decisión de usarlo, para poder realizar
junto a los oftalmólogos, pruebas y ajustes en tiempo real según sus
indicaciones de la manera más rápida posible.\\
Una aprendido lo básico mediante los tutoriales de la página oficial,
dediqué mi tiempo de estudio a la bibliotecas \emph{SimpleCV} primero,
y posteriormente \emph{OpenCV}. Aunque tienen muy buena documentación
y están muy bien diseñadas, es necesario dedicarlas muchas horas para
comprender en profundidad todas las técnicas que posteriormente
tuvimos que aplicar además de generar nuestros propios algoritmos y
técnicas auxiliares (como la extracción de parte de la tomografía y su
posterior corrección de inclinación del corte con respecto a la
horizontal) para conseguir nuestros objetivos.\\
De los dos algoritmos propuesto, me centré en realizar su
implementación y preprocesado previo más que en la teoría de con qué
parámetros aplicar que estudió previamente mi compañero Daniel. Una
vez que estuvieron ambos algoritmos estuvieron correctamente
desarrollados, me centré en reducir al máximo posible todos los
errores de precisión al igual que por otra parte aumentar la
tolerancia al ruido en las tomografías con peor calidad.\\
Finalmente para mejorar la comunicación con los algoritmos, realicé
una envoltura del módulo \emph{argparse} de la biblioteca estándar de
\emph{Python} para por una parte hacer lo más simple para los
oftalmólogos el procesado de archivos y carpetas y por otra parte
añadir un modo de generación de una imagen por cada paso realizado
para poder depurar más rápido y eficientemente los algoritmos y sus
imprecisiones.

\subsection{Memoria}
Otra parte muy importante del trabajo, fue la realización de la
memorial. La cual, por decisión unánime debido a la envergadura del
documento, índices, partes, fórmulas, imágenes y código a incluir, se
decidió escribir en \LaTeX.\\
Para ello, me encargué tanto de la generación desde $0$ de este
documento además su formato, tanto para su uso digital como su
adaptación para impresión.\\
Gran parte del tiempo que he dedicado en la elaboración de la memoria,
lo destiné a la búsqueda y uso de paquetes del repositorio
\emph{CTAN}.\\
Por una parte, para hacer el trabajo de la generación del documento
más simple y cómoda, añadí además de numerosos paquetes e
instrucciones personalizadas (para reducir el código a escribir) y
otras para realizar ciertas acciones que \LaTeX\ no proporciona de
manera predeterminada (bloques de código Python, términos en
\emph{Siglas} y \emph{Glosario} simultáneamente o aumentar la
profundidad de enumeración de apartados entre otras muchas
elaboraciones). Además un archivo \emph{Makefile} para automatizar con
\emph{latexmk} por completo todo el proceso de compilación.\\
Por otra parte, para hacer que fuera lo más visual y atractiva posible
mediante, referencias (paquete \emph{hyperref}), coloreado del código
(paquete \emph{minted}), cambio del tipo de letra, etc.\\
Finalmente, me centré en escribir los capítulos más técnicos de la
memoria así como de describir y dibujar todas las fórmulas
matemáticas.

\subsection{Conocimientos adquiridos}
Los conocimientos adquiridos han sido muy variados. Nunca antes había
participado en una investigación, por lo que tuve que aprender a
desenvolverme con los oftalmólogos en un entorno hospitalario. Ni
tampoco había realizado un documento tan técnico, extenso y con tantos
enlaces bibliográficos.\\
Además, tuve que cambiar mi manera de pensar, primero, adaptarme a
todas las fórmulas y abstracciones matemáticas de las técnicas de
\emph{Visión computarizada} para posteriormente centrarme en los
resultados visuales obtenidos tras aplicarse así como también poder
discernir los pasos siguientes o cambios para obtener mejores
resultados.\\
Como conclusión, se puede apreciar a simple vista que todo el tiempo
dedicado a este trabajo por mi parte ha sido muy satisfactorio y
fructífero. Me ha proporcionado una visión y manera de trabajar con
problemas complejos e interdisciplinares. Proponer y elaborar
objetivos a partir de reuniones y estudios previo de viabilidad así
como también sugerir y llevar a cabo soluciones mediante una primera
aproximación abstracta y matemática a su desarrollo e implementación
concreta mediante código preciso y eficiente.
