\section*{Resumen}
Durante la realización de un diagnóstico médico se genera una gran
cantidad de información. En muchas ocasiones esta información cobra
forma de imágenes: radiografías, tomografías, resonancias magnéticas,
ecografías, etc. Esto crea la necesidad de mejorar las técnicas de
estudio de estas imágenes, facilitando y automatizando su
interpretación, de forma que el diagnóstico sea más rápido y
exacto. Así se pretende aumentar la precisión en la detección y el
seguimiento de enfermedades.\\
Este proyecto se centra en el estudio de un tipo de imagen concreto: las tomografías
generadas por una máquina de Tomografía de Coherencia Óptica. Esta
máquina mediante la reflexión de ondas de luz crea una representación
visual de los tejidos de la parte interna del ojo (Para más detalle
consultar el primer capítulo de la parte III). Más exactamente, el
estudio llevado a cabo se centra en tomografías de la
\gls{papila-optica} o cabeza del nervio óptico y la \gls{coroides}:
\begin{itemize}
\item Sobre la \gls{papila-optica} se necesita definir, marcar y medir
  el tamaño de los poros de la lámina cribosa, que están siendo objeto
  de estudio debido a su posible relación con la aparición del
  \gls{glaucoma}.
\item Sobre la \gls{coroides} se necesita definir, marcar y medir el
  grosor de la misma debido a la relación existente entre su grosor y
  diversas enfermedades, entre ellas, la \emph{\gls{uveitis}}.
\end{itemize}
El objetivo es hacer un tratamiento lo más automatizado posible de estas imágenes utilizando algoritmos de \emph{Visión Computerizada}
mediante su implementación con distintas bibliotecas de \emph{software
  libre} tanto propias como de terceros que se explicarán más
adelante. La razón de crear y usar exclusivamente \emph{software
  libre} surge, por una parte, de no depender del software costosísimo
patentado y protegido de las máquinas, así como de la necesidad de
conocimiento profundo y transparente de su funcionamiento, debido a la
aparición de nuevas necesidades muy precisas de los oftalmólogos,
tanto en las tareas más rutinarias como en las más pioneras. Estas
necesidades siempre estarán un paso por delante de las facilidades y
adaptaciones proporcionadas por las actualizaciones propietarias de
las grandes compañías, siendo éstas además difícilmente costeables.
\newpage
\section*{Abstract}

