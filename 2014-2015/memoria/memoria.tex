\documentclass[11pt,a4paper,twoside]{book}
\usepackage[T1]{fontenc}
\usepackage[utf8]{inputenc}
\usepackage[spanish]{babel}
% \usepackage[spanish]{translator}
% Permitir que todas las listas enlacen con su referencia
\usepackage[pdftex]{hyperref}
% Quitar el color distintivo a los enlaces
\hypersetup{hidelinks=true}

% https://www.ctan.org/pkg/glossaries
\usepackage[acronym]{glossaries}
\makeglossaries
% Añadir los términos del glosario
% http://tex.stackexchange.com/a/8951
% http://osl.ugr.es/CTAN/macros/latex/contrib/glossaries/glossariesbegin.pdf

% definir el término TCO
\newglossaryentry{tcog}{name={TCO},
    description={FALTA LA DESCRIPCIÓN DEL TÉRMINO}}

% ECMI acrónimo y glosario
\newdualentry{ECMI} % etiqueta
  {ECMI}            % abreviatura
  {European Consortium for Mathematics in Industry}  % forma larga
  {Descripción} % descripción

% https://www.ctan.org/pkg/bibtex
% Traducir bibtex con babel
\usepackage{babelbib}
\usepackage{cite}
% No descomentar hasta usarlos
%\usepackage{amsmath}
%\usepackage{amssymb}
\usepackage{graphicx}

\begin{document}
% http://stackoverflow.com/a/22155949
% Eliminar todas las páginas vacías
\let\cleardoublepage\clearpage

% Portada
% Incluirá una portada normalizada conteniendo la siguiente información: título, autores,
% profesor director, codirector si es el caso, curso académico e identificación de la
% asignatura (Trabajo de fin de grado del Grado en nombre del grado correspondiente,
% Facultad de Informática, Universidad Complutense de Madrid).Los datos referentes al
% título y director (y codirector en su caso) deben corresponder a los publicados en la
% lista indicada en los puntos 8 y 9 de la sección III de esta
% normativa.

% Título: Visión Computerizada, aplicaciones reales en Medicina
% Director: Carlos Gregorio
% Alumnos: Miguel Madrid Mencía, Daniel Arnao Rodríguez
\title{Visión Computerizada: Aplicaciones Reales en Medicina}

\author{Daniel Arnao Rodríguez \and Miguel Madrid Mencía }

\date{2014 -2015}

\maketitle

% Contendrá al menos un índice; un resumen y una lista de no más de 10 palabras clave
% para su búsqueda bibliográfica (ambos en castellano e inglés); una introducción con los
% antecedentes, objetivos y plan de trabajo; los resultados y una discusión crítica y
% razonada de los mismos, con sus conclusiones. En particular la memoria debe incluir la
% descripción detallada de la propuesta hardware/software realizada. También se
% incluirá una relación de la bibliografía empleada en la elaboración de la memoria.

% números romanos
\frontmatter

% Dedicatorias
\chapter*{Miguel}

\chapter*{Daniel}
A mi familia y mis amigos, que siempre
han estado ahí dándome apoyo.
% Agradecimientos
\chapter*{Agradecimientos}
En primer lugar agradecer a Carlos Gregorio Rodríguez,
director de este proyecto, la oportunidad que nos ha
brindado de llevar a cabo un trabajo de estas características,
así como toda la ayuda que nos ha proporcionado. Sin él
no hubiera sido posible llegar a buen puerto. Muchas gracias Carlos.\\

También al equipo de oftalmólogos del \emph{Hospital 12 de Octubre}
de Madrid que ha colaborado con nosotros, proporcionándonos
las imágenes de estudio y el conocimiento necesario para
procesarlas. Nuestro más sincero agradecimiento para todos ellos,
y especialmente a Esperanza Gutiérrez Díaz y Javier Sambricio García. \\

Por último, agradecer a toda la gente que de manera indirecta
ha ayudado a que este proyecto salga adelante: pacientes,
amigos, compañeros, etc. Gracias a todos.
% Palabras clave
\chapter*{Lista de palabras clave}
\textbf{Visión Computarizada, Tomografía Coherencia Óptica, TCO,
  espesor coroides, uveítis, poros papila, glaucoma, OpenCV,
  \mbox{Python}, SimpleCV.}
\chapter*{Keywords List}
\textbf{Computer Vision, Optical Coherence Tomography, OCT, choroid
  thickness, uveitis, papilla pores, glaucoma, OpenCV, Python,
  SimpleCV.}

% Resumen
\section*{Resumen}
Durante la realización de un diagnóstico médico se genera una gran
cantidad de información. En muchas ocasiones esta información cobra
forma de imágenes \emph{\citep*[1. The Analysis of Medical Images,
  2. Digital Image Acquisition]{toennies2012guide}}: radiografías,
tomografías, resonancias magnéticas, ecografías, etc. Esto crea la
necesidad de mejorar las técnicas de estudio de estas imágenes
\emph{\citep*[4. Image Enhancement]{toennies2012guide}}, facilitando y
automatizando su interpretación, de forma que el diagnóstico sea más
rápido y exacto. Así se pretende aumentar la precisión en la detección
y el seguimiento de enfermedades.\\
Este proyecto se centra en el estudio de un tipo de imagen concreto:
las tomografías generadas por una máquina de Tomografía de Coherencia
Óptica. Esta máquina mediante la reflexión de ondas de luz crea una
representación visual de los tejidos de la parte interna del globo
ocular (para más detalle consultar el primer capítulo de la parte
III). Más exactamente, el estudio llevado a cabo se centra en
tomografías de la \gls{papila-optica} o cabeza del nervio óptico y la
\gls{coroides}:
\begin{itemize}
\item Sobre la \gls{papila-optica} se necesita definir, marcar y medir
  el tamaño de los poros de la \emph{\gls{lamina-cribosa}}, que están
  siendo objeto de estudio debido a su posible relación con la
  aparición del \gls{glaucoma}.
\item Sobre la \gls{coroides} se necesita definir, marcar y medir el
  grosor de la misma debido a la relación existente entre su grosor y
  diversas enfermedades, entre ellas, la \emph{\gls{uveitis}}.
\end{itemize}
El objetivo es hacer un tratamiento lo más automatizado posible de
estas imágenes utilizando algoritmos de \emph{Visión Computarizada}
mediante su implementación con distintas bibliotecas de \emph{software
  libre} tanto propias como de terceros que se explicarán más
adelante. La razón de crear y usar exclusivamente \emph{software
  libre} surge, por una parte, de no depender del software costosísimo
patentado y protegido de las máquinas, así como de la necesidad de
conocimiento profundo y transparente de su funcionamiento, debido a la
aparición de nuevas necesidades muy precisas de los oftalmólogos,
tanto en las tareas más rutinarias como en las más pioneras. Estas
necesidades siempre estarán un paso por delante de las facilidades y
adaptaciones proporcionadas por las actualizaciones propietarias de
las grandes compañías, siendo éstas además difícilmente costeables.

\newpage

\section*{Abstract}
During the execution of medical diagnosis a lot of information is
generated.  Many times, this information takes the form of images
\emph{\citep*[1. The Analysis of Medical Images, 2. Digital Image
  Acquisition]{toennies2012guide}}: radiographies, tomographies,
magnetic resonances, ecographies, etc.  This creates the need to
improve the study techniques which are used for these images
\emph{\citep*[4. Image Enhancement]{toennies2012guide}}, making their
interpretation easier and automatic in order to get a fast and exact
diagnosis.Being the objetive of this increasing the
precission while detecting and monitoring the illness.\\
This project focuses on the study of a particular type of image: those
generated by optical coherence tomography machines, which creates a
visual representation of the inner leyers of the eye by the reflection
of light waves (for more information consult the first chapter of Part
III). More precisely, the study objective is focused on scans of the
optic disc or optic nerve head and the choroid:
\begin{itemize}
\item On the optic nerve it is necessary to define, mark and measure
  the size of the pores of the \emph{lamina cribrosa}, which are being
  studied because of their potential relation to the onset of \emph{glaucoma}.
\item On the choroid it is necessary to define, mark and measure its
  thickness because of the relation between it and various diseases, including uveitis.
\end{itemize}
The goal is to make the treatment of the images as automatic as
possible using Computer Vision algorithms through its implementation
with various free software libraries. Some of the libraries that have
been used were self-implemented and others made by third parts as
free-software libraries. These libraries will be detailed later. The
reason for creating and using free software exclusively comes, on the
one hand, from not depending on the very expensive patented and protected
software and, on the other hand, the need for thorough transparent
knowledge of its operation. This will respond to the new needs of the
ophthalmologists in the routine tasks and in the pioneer ones. These needs 
are always one step ahead of the
facilities and adaptations proprietary updates provided by big
companies, which are also hardly affordable.

% Lista de acrónimos
\printglossary[type=\acronymtype]

\tableofcontents

% Índice
% Parte 1 Antecedentes
% 1. Introducción
%    1.1 Problema
%    1.2 Objetivos
% 2  Tomografías de coherencia Óptica
%    2.1 Máquina
%    2.2 Imágen
%
% Parte 2 Visión computerizada
% 2. Visión computerizada
%    2.0 Introducción
%    2.1 Estudio previo
%        Investigaciones o programas parecidos, mencionar http://ecmiindmath.org/2015/04/07/optical-modelling-of-the-human-retina/
%    2.2 Conocimientos necesarios
%    2.3 Tecnología utilizada
%        OpenCV, SimpleCV, numpy
%    2.4 Tecnología descartada
%        Matlab
% Parte 3 Investigación
% 3. Detección de poros
%
% 4. Espesor coroides
%    4.1 Definición
%    4.2 Dificultades
%    4.3 Detección
%    4.4 Identificación
%
% 5. Propuesta software
% Parte 5 Conclusiones
% 5. Conclusiones
%
% 6. Futuro
%
% Apéndice - Contribuciones

% Lista de figuras

% números arábigos
\mainmatter

\include{memoria/capitulos/parte_I/parte_I}
\chapter{Introducción}
\section{Problema}
Este proyecto nace de la necesidad que tienen los médicos de mejorar y
automatizar el estudio de las imágenes obtenidas de los pacientes. Ya
sea porque las máquinas que generan dichas imágenes carecen de las
técnicas requeridas por los médicos o por el intento de automatizar el
seguimiento rutinario de la evolución de una enfermedad a partir de su
historial de imágenes en el tiempo. \\
Por ser el problema tan amplio y poco concreto, nos hemos centrado en
colaborar con oftalmólogos del \emph{Hospital Universitario 12 de
  Octubre} en el estudio imágenes de \gls{tcoa} tomadas para seguir la
evolución dos enfermedades, la \emph{uveitis} y el \emph{glaucoma}

\section{Objetivos}
Aunque el problema a primera vista es demasiado amplio e inabarcable
los objetivos están muy claros por los oftalmólogos. De hecho, les
pedimos que nos marcaran a mano lo que necesitaban estudiar de las
\gls{tcoa}. Así conseguimos enumerar los siguientes objetivos:
\begin{itemize}
\item Obtener imágenes tan fieles como sea posible a las marcadas
  manualmente.
\item Conseguir automatizar al máximo dicho proceso o al menos
  facilitarlo para poderse realizar en el menor número de pasos posibles.
\item Proporcionar los conocimientos obtenidos de los programas, datos,
  medidas\ldots\ para mejorar el entendimiento tanto de las \gls{tcoa}
  como de las enfermedades estudiadas.
\item Aprender a desenvolvernos en un entorno completamente ajeno al
  nuestro con un montón de técnicas para nosotros desconocidas para alcanzar
  resultados aplicables y útiles en el  mundo real y más concretamente
  en entornos hospitalarios.
\item Comprender en su totalidad el problema, los objetivos y todas
  las soluciones propuestas para en un futuro proponer mejoras,
  atrevernos con nuevos problemas y desarrollar técnicas más complejas
  para proponernos objetivos más difíciles.
\end{itemize}

\chapter{Tomografías de Coherencia Óptica}
\section{¿Qué son?}
\section{Máquina \glsentrytext{tcog}}
Una máquina \gls{tcoa} se utiliza para la exploración en tiempo real, sin ser
invasiva e indolora, para obtener con resolución micrométrica imágenes
de estructuras de tejidos vivos.Con el paso del tiempo se ha hecho
insustituible en el diagnóstico y control de numerosas patologías
oculares, así como también la mejora de su rapidez y precisión.
\subsection{Base teórica}
La máquina \gls{tcoa} usa una técnica análoga a las máquina de ecografías
pero sustituyendo los ultrasonidos por luz. La máquina mide la
latencia o retardo e intensidad de la onda reflejada tras incidir en
un tejido. La ventaja de una máquina \gls{tcoa} frente a una de ecografías es
la velocidad de la luz, muy superior a la velocidad de los
ultrasonidos requiriendo un sistema de medición indirecto.
\subsection{Propiedades ópticas de los tejidos}
La máquina \gls{tcoa} es capaz de medir y representar los distintos
tipos de tejidos en escala de grises o colores según su reflectividad:
\begin{description}
\item[Reflectividad alta:] sangre, exudados lipídicos, epitelio pigmentario,
coriocapilar o capas de fibras nerviosas perpendiculares al haz de
luz. Representados con colores cálidos como rojo y blanco.
\item[Reflectividad media:] capas retinianas de la limitante interna a la
plexiforme externa. Representadas con colores verdes y amarillos
\item[Reflectividad baja:] zonas de edema, vítreo y contenido seroso,
dispuestos en paralelo al haz de luz. Representados con colores fríos
como azul y negro.
\end{description}
\section{Imágenes \glsentrytext{tcog}}

\part{Visión computerizada}
\chapter{Técnicas de visión computerizada}

\include{memoria/capitulos/parte_III/parte_III}

\include{memoria/capitulos/parte_IV/parte_IV}


\appendix

% antes del final, para no numerar
\backmatter

% Glosario 
\printglossary[type=main]

% Bibliografía
% Añadir la bibliografía
\bibliography{memoria/bibliografia/bibliografia}
\bibliographystyle{babplain}

\end{document}
